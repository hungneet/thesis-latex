\chapter{SYSTEM IMPLEMENTATION}

With all the system design and system analysis completed in the previous semester, we now embark on the implementation phase of our project. This chapter details the process of bringing our design to life, outlining the key features of our system and the methodologies employed to implement them. Through careful planning and execution, we aim to translate our theoretical framework into a functional and efficient system. Below are some of the critical features of our system and the approaches we used to implement them.


\section{Caching Mechanism}

To reduce the amount of time it takes to access frequently accessed data
and increase the general system performance, we will apply caching data
in our system. The applied caching strategy is called \emph{aside
    caching} \emph{(Figure 5.1).} The detailed caching process is as follows:

- Step 1: The application will check whether the cache has the data it
needs or not.

- Step 2: If the cache has the data the application needs, the process
will end. If the cache has no data, we will go to the next step.

- Step 3: When the cache does not contain the data that the application
needs, the application will go to the database to get the data

- Step 4: The application will save the data retrieved from the database
to the cache, then it will continue its work.

\begin{figure}[H]
    \centering
    \includegraphics[width=0.8\textwidth]{Figures/caching_strat.png}
    \caption{Caching Mechanism}
\end{figure}

\section{Advanced search strategy}

Querying data that requires applying many filters (pet profiles, etc)
from the SQL database is extremely time-consuming. Note that only data
that is used repeatedly and is not too large in size can be stored
temporarily in the cache server. Therefore, applying a search strategy
to optimize querying the above type of data is necessary.

\emph{Figure 5.2} shows the steps of syncing data from the SQL database
to the Elastic-search database.

\begin{itemize}
    \item
          Whenever the application does an action on a record of the SQL
          database (create, update, delete), a new record (which includes the ID
          of the record taken the action, and the action), from now we call the
          async record, is created.
    \item
          At the end of the application process, a background job is triggered
          to query all the async records and send all the records that they
          point to onto the Elastic-search database. Therefore the data of the
          Elastic-search database is always synchronized with the SQL database.
    \item
          Note that the application only queries data from the Elastic-search
          database for tables that were defined before.
\end{itemize}

\begin{figure}[H]
    \centering
    \includegraphics[width=0.8\textwidth]{Figures/search_strat.png}
    \caption{Search strategy}
\end{figure}

\section{Online payment}

In our system, we have integrated the PayPal Braintree Gateway to handle online payments securely and efficiently. Braintree is a full-stack payment platform that offers a seamless payment experience, supporting various payment methods, including credit cards, PayPal, and other digital wallets. Its robust infrastructure and comprehensive SDKs for both client and server sides make it an ideal choice for our system's payment processing needs.

\begin{figure}[H]
    \centering
    \includegraphics[width=0.7\textwidth]{Figures/payment-strat.png}
    \caption{Payment process}
\end{figure}

\emph{Figure 5.3} breaks down how the payment process is implemented:

- Step 1:
The front-end requests a client token from our server and initializes the Braintree client. This token is essential for securely communicating between the client and the Braintree server.

- Step 2:
Our server generates a client token using Braintree's tools and sends it back to the client. This token allows the front-end to securely handle customer payment information.

- Step 3:
The customer enters their payment information on the front-end. The Braintree client communicates this information to Braintree and returns a payment method nonce, a secure reference representing the payment details.

- Step 4:
The front-end sends the payment method nonce to our server. This nonce is a secure way to pass payment information without exposing sensitive data.

- Step 5:
The server receives the payment method nonce and uses Braintree's tools to process the payment. The server then communicates with Braintree to complete the transaction securely.

By utilizing PayPal Braintree Gateway, we ensure a secure and streamlined payment process, enhancing the user experience and maintaining high security standards in handling sensitive payment information.