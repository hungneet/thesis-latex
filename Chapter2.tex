
\chapter{SYSTEM ANAYLYSIS}
\section{Market Survey}
To comprehend the scope and orientation of the system, it’s essential to conduct a market survey of similar systems. Pet adoption systems, each with its unique features and benefits, come in various forms. In Vietnam, however, most organizations in this field operate through social network pages like Facebook and Instagram, rather than standalone software systems. Notably, the only pet adoption system that stands out from the rest is Hanoi Pet Adoption, which warrants further analysis and evaluation.

\subsection{Overview}
The Hanoi Pet Adoption (HPA) system, a web application based in Hanoi, Vietnam, serves as a platform for individuals to find and adopt rescued animals. The primary objective of this system is to aid the HPA group in rescuing and caring for animals, and subsequently finding them loving and responsible homes.

HPA has the following main features:
\begin{itemize}
  \item \textit{Pet profiles:} The system showcases profiles of animals ready for adoption. These profiles include details such as photos, names, genders, ages, colors, and sterilization statuses.
  \item \textit{User Preferences:} Users can filter animals based on their preferences, including gender, age, color, and sterilization status.
  \item \textit{Volunteer Contact Information:} The system provides the contact information of volunteers responsible for each animal, enabling users to inquire further about their animal of interest.
  \item \textit{Adoption Process Guidance:} The system guides users through the adoption process, which includes an interview, a contract, and a fee.
\end{itemize}

\subsection{Evaluation}
HPA provides a comprehensive and effective pet adoption support process. However, its limitation lies in its exclusivity - only system administrators of the organization are permitted to post pet profiles onto the system.

The absence of a messaging system in the current platform can create challenges in the pet adoption process. This lack of direct communication makes it difficult for system administrators and users to contact each other, particularly when issues arise. A messaging system would facilitate smoother interactions and problem resolution, enhancing the overall user experience.

Currently, the system requires manual input for posting, filtering, and searching functions. This can pose challenges for individual users, potentially reducing the efficiency of posting and searching, and negatively impacting the user experience.

Lastly, a significant feature that is absent in the system is a notification system. Such a system would alert users when pets that meet their criteria are posted, ensuring that users are promptly informed about potential matches. This feature could greatly enhance the user experience and effectiveness of the pet adoption process.

\subsection{Potential Improvements}
After a thorough review of the strengths and weaknesses of existing systems, we have identified several potential enhancements that could significantly improve the user experience and functionality of the system.

Firstly, we propose to \textbf{expand the HPA system}. Currently, the pets are only posted by system administrators. However, SGT aims to allow individuals wishing to surrender their pets to proactively post about their pets for adoption. This expansion would not only increase the number of pets available for adoption but also provide a platform for pet owners and organizations to share valuable information and updates about their pets.

Secondly, we suggest the implementation of \textbf{interactive pet profiles}. This feature would allow users to visit and interact with pet profiles and posts. This interaction could include liking, commenting, or sharing a pet’s profile.

Next, we recommend \textbf{applying Artificial Intelligence (AI) technology to enhance the posting and searching} functionality of the system. With this feature, users could simply upload a pet image, and the AI would automatically populate most of the input fields, such as breed, and color. This would not only simplify the process of posting a pet profile but also increase the accuracy and consistency of the pet information in the system.

In addition, we propose the implementation of a \textbf{notification system}. This system would alert users when a pet that matches their preferences becomes available on the system. User criteria for filtering pet profiles(such as breed, age, or size) will be referenced, and then users can receive notifications when a matching pet is posted. This would ensure that users don’t miss out on potential matches and can act quickly to adopt their desired pet.

Lastly, we suggest establishing a \textbf{messaging system} that facilitates communication among pet adopters, pet owners, and system administrators. This system would allow users to ask questions, clarify information, or arrange meetings, enhancing the overall user experience and streamlining the adoption process. By providing a platform for direct communication, we can foster a sense of community among users and promote open and transparent discussions about pet adoption.

In conclusion, these enhancements aim to make the system more user-friendly, efficient, and effective in connecting pets with potential adopters. We believe that by implementing these changes, we can take a significant step toward our goal of finding a loving and suitable home for every pet.

\section{Stakeholders}
Stakeholders for a project focused on creating a pet adoption platform in Vietnam would include a diverse range of individuals, organizations, and groups with an interest in or influence over the project. Here are some key stakeholders:
\begin{itemize}
  \item \textit{Administrator:} The administrator is responsible for managing and overseeing the platform's operations. They have control over user accounts, content moderation, and the overall functionality of the website.
  \item \textit{Pet Adopter:} Pet adopters are individuals looking to provide a loving home to pets available for adoption. They may already own pets or be first-time pet owners. Pet adopters interact with pet owners and adoption agencies through messaging and inquiries about available pets. They can also share their adoption stories and experiences with the platform's community.
  \item \textit{Pet Owner:} Pet owners are individuals who currently own and care for pets, including dogs, cats, or other animals. Or those, for various reasons, have decided to give their pets to new homes. Reasons may include relocation, financial constraints, allergies, or changes in life circumstances.
  \item \textit{Guest:} Guests are individuals who visit the platform without creating an account or logging in. They have limited access to platform features and content. Guests can browse public content, view pet listings, read provided blogs, and explore the platform's resources. They can also choose to create an account to access more features.
  \item \textit{Pet Adoption Agencies and Shelters:} Organizations involved in pet adoption, rescue, and rehoming would benefit from the platform as it could help them find suitable homes for abandoned or rescued animals. They might also contribute content and listings.
  \item \textit{Pet-Related Businesses:} Pet stores, pet food suppliers, grooming salons, veterinary clinics, and other businesses in the pet industry have a stake in the project. They could use the platform to advertise their products and services to a targeted audience.
  \item \textit{Veterinarians:} Veterinarians play a crucial role in pet healthcare. They might use the platform to provide information, answer questions, and offer telehealth services. Their expertise can be valuable to pet owners.
  \item \textit{Investors and Funders:} Individuals or organizations providing funding or investment for the development and scaling of the platform are stakeholders with a financial interest.
\end{itemize}

Understanding and engaging with these stakeholders will be important for the success and sustainability of the project. Each stakeholder group may have different needs, interests, and concerns that should be addressed during the project's planning and execution.

\section{Requirements eliciation}
\subsection{Functional Requirements}

\begin{longtblr}[
    caption = {Functional Requirements},
    label = {tblr:func_req},
  ]{
    vline{1-3} = {-}{},
    hline{-} = {1-2}{},
    colspec={X[2,l] X[5, l]},
  }
  \textbf{Group}                                & \textbf{Requirements}                                                                                                                                                                                                                                                                                                                                                                                                                                                                                                                                                                      &  &  \\
  \textbf{Authentication and authorization}     & {
    -~~~~~~~
    Users can create new accounts by providing personal information.
    \\-~~~~~~~
    Users can log into the system using Google accounts.
    \\-~~~~~~~
    Users can log into the system by
    providing a registered email and password.
    \\-~~~~~~~
    Users can send applications to the system to update their accounts to
    Organization accounts.
    \\-~~~~~~~
    Admins can verify Organizations’ registration and upgrade users’
    accounts.
    \\-~~~~~~~
    Users can reset or recover their account passwords.
    }                                                          &  &  \\
  \textbf{Pet profile management}               & {
    -~~~~~~~
    Users can create pet profiles by giving details such as name, species,
    breed, age, color, health status, sterilization status, pictures, and videos
    of pets.
    \\-~~~~~~~
    Users can provide information about breed, species, and color by
    uploading pets’ pictures.
    \\-~~~~~~~
    Pet photos and text inputs must be sensitive-validated before
    uploading.
    \\-~~~~~~~
    Users can edit their pet profiles.
    \\-~~~~~~~
    Users can publish, hide, or remove their pet profiles.
    \\-~~~~~~~
    Admins can remove or hide any pet profiles of the system.
    } &  &  \\
  \textbf{Pet profile filter and view}          & {
    -~~~~~~~
    Users can search for pets by providing filtering options such as name,
    species, breed, age, color, health status, and sterilization status.
    \\-~~~~~~~
    Users can search for pets by providing pets’ pictures.
    \\-~~~~~~~
    Users can view pet profiles with detailed information such as name,
    species, breed, age, color, health status, sterilization status, pictures,
    and videos of pets
    }                                                                                                                                                                &  &  \\
  \textbf{Pet adoption}                         & {
    -~~~~~~~
    Users can submit applications to adopt pets on the system.
    \\-~~~~~~~
    Users can view, decline, or accept adoption applications sent to their
    pets.
    }                                                                                                                                                                                                                                                                                                                                                                                                                &  &  \\
  \textbf{User interaction and engagement}      & {-~~~~~~~
  Pet
  Adopters can send messages to Pet Owners, Organizations, and Admins.
  \\-~~~~~~~PetAdopters have to update their pets’ status every seven days[1]~ from the date they received the petsby uploading posts in the adopt status section of the pet profile.\\-~~~~~~~
  Pet
  Adopters can set their posts public for all users or just Admins, and Pet
  Owners.
  \\-~~~~~~~
  Admins
  and Pet Owners can view and interact with posts from Pet Adopters.
  \\-~ ~ ~ ~ All users of the system can view and interact
  with posts from Pet Adopters if the posts are public.~ ~ ~~}        &  &  \\
  \textbf{Blog management}                      & {
    -~~~~~~~
    Admins and Organizations can create and publish blogs.
    \\-~~~~~~~
    Organizations can apply advertisements on their blogs.
    \\-~~~~~~~
    Organizations can edit or remove their blogs.
    \\-~~~~~~~
    Admins can hide or remove any blogs of the system.
    }                                                                                                                                                                                                                                                                                                             &  &  \\
  \textbf{Advertisement and payment}            & {
    -~~~~~~~
    Organizations can choose advertisement duration with different prices.
    \\-~~~~~~~
    Organizations can pay advertisement fees through an online banking
    service.
    \\-~~~~~~~
    Admins can receive advertisement fees through an online banking
    service.
    \\-~~~~~~~
    System provides admins and organizations information like payment
    status and invoices.
    }                                                                                                                                                                                                   &  &  \\
  \textbf{Notifications}                        & {
    -~~~~~~~
    Users can receive notifications on new pet adoption applications.
    \\-~~~~~~~
    Users can receive notifications on new pets’ status posts uploaded.
    \\-~~~~~~~
    Users can receive notifications on new pet profiles matching their
    criteria.
    }                                                                                                                                                                                                                                                                                                                      &  &  \\
  \textbf{System statistics for administrators} & -~~~~~~~
    Admins can get statistics about systems usages, such as the number of
    pet profiles, adopted pets; number of users and Organizations; number of
    blogs, and the total amount of advertisement fee.                                                                                                                                                                                                                                                                                                                                                                            &  &  
\end{longtblr}


