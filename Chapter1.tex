
\chapter{Giới thiệu}
\section{Content}
In recent years, Vietnam, and Ho Chi Minh City, in particular, have seen a significant surge in the demand for pet adoption. This growing trend underscores the increasing necessity for organizations involved in pet adoption to enhance their systems and operational procedures. This will ensure they can effectively meet the escalating demand and continue to provide excellent service to those seeking to adopt pets.

Established in 2015, Saigon Time (SGT) has earned a reputation as one of the leading stray pet organizations in Ho Chi Minh City. The organization has been instrumental in providing care and treatment for numerous abandoned dogs and cats, successfully finding them new homes. Like many pet rescue and adoption organizations in the city, SGT primarily operates through a Facebook fan page. Recognizing the need for more effective management and a professional pet adoption process, as well as the desire to expand the organization’s operations, SGT has decided to develop a dedicated software system.

\section{Problems}

As previously noted, Saigon Time operates exclusively on Facebook. However, with the growing user base and the organization’s desire to expand its operational capabilities, several challenges have emerged:
\begin{itemize}
    \item User Experience: Users are finding it difficult to search for and identify suitable pets for adoption. The current process does not effectively accommodate the extensive information about each pet that needs to be considered before adoption.
    \item Pet Giving Process: Currently, the pets in SGT’s system are those physically present at their operating location. In the future, SGT aims to allow individuals wishing to give their pets to proactively post about their pets for adoption. This development necessitates the implementation of a system capable of managing the adoption process between users.
    \item System Management: The increasing number of users and pets has made it challenging for SGT administrators to manage the system. Information such as the number of pets, number of users, etc., is currently managed using office tools. This situation underscores the need for a centralized management solution for pet posting, adoption, data management, and system operations.
\end{itemize}

\section{Goals}

A software system can serve as a solution to the mentioned issues by establishing a platform that bridges the gap between pet shelters, rescues, pet owners, and potential adopters. This system can be tailored to assist individuals in finding pets that align with their preferences and lifestyle, offering insights into the pet’s health, temperament, and background. It can also simplify the adoption process by enabling users to submit adoption applications online, schedule appointments to meet pets, and finalize the adoption process digitally.

Furthermore, the software system can enhance the efficiency of SGT administrators in managing their operations. It can offer tools for handling pet records, monitoring adoptions, and tracking system usage. Additionally, it can assist shelters, rescues, and other businesses in promoting their services.

Lastly, the system is designed to transform SGT into a hub where pet enthusiasts can access information and gain knowledge about pet care.

\section{Scope}

This specialized project aims to develop a comprehensive software system that will streamline the operations of the SGT organization, moving away from its current reliance on a Facebook page. The proposed software system will consist of:
\begin{itemize}
    \item 	A web application for users who are interested in adopting or giving away pets, as well as for organizations that wish to collaborate with SGT.
    \item	A web application specifically designed for SGT administrators to effectively manage the entire system.
    \item 	Supporting programs and services to ensure the accuracy and seamless operation of the aforementioned applications.
\end{itemize}

\section{Report structure}
Our specialized project report will be structured into the following key chapters:
\begin{enumerate}
    \item \textbf{Introduction}: This chapter will provide essential background information on the project topic and clearly state the objectives of the project.
    \item \textbf{System Analysis}: This chapter will concentrate on the collection and analysis of business requirements. It will employ a range of techniques to translate these requirements from a common language into a format suitable for system design and implementation.
    \item \textbf{System Design}: This chapter will present detailed designs and methodologies to fulfill the requirements established in the system analysis phase.
    \item \textbf{System Implementation}: This final chapter will outline the process involved in implementing the system functions, providing a comprehensive view of how the system was brought to life.
\end{enumerate}


Each chapter will provide a thorough exploration of its respective topic, ensuring a comprehensive understanding of the project.
