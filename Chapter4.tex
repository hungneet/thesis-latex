\chapter{DEVELOPMENT TECH STACK}

\section{Front-end}


In this section, we will dive into Next.js and Tailwind CSS. Next.js is
a React framework with features like server-side rendering, while
Tailwind CSS is a utility-first framework for streamlined styling
through pre-designed classes. These tools play a key role in creating
efficient and scalable front-end experiences.


\subsection{NextJS}

Next.js, a robust React framework, tackles challenges in traditional
client-side rendering. It utilizes server-side rendering and static site
generation to address accessibility, security, page loading times, and
Search engine optimization (SEO) concerns. Popular in the React
ecosystem, Next.js optimizes the user experience by pre-rendering pages
on the server. Advantages include enhanced performance with server-side
rendering, seamless React integration for dynamic interfaces, static
site generation for optimal website performance, intuitive file
system-based routing, automatic code splitting, flexible styling
approaches, and compatibility with TypeScript for improved code quality.


\subsection{Tailwind CSS}

Tailwind CSS, a utility-first framework, transforms web styling by
providing an extensive set of classes directly within HTML. This
accelerates development by offering granular control over layout, color,
spacing, and more. Unlike traditional CSS, Tailwind\textquotesingle s
unique approach eliminates the need for external files, enabling
seamless customization and efficient creation of visually appealing and
responsive interfaces. Its utility-first approach minimizes unused CSS,
optimizing performance, and its low learning curve, active community,
and integration flexibility make it an efficient choice for building
well-styled and maintainable web applications.


\section{Back-end}
\subsection{ASP.NET core platform}

ASP.NET Core, an open-source framework, excels in developing modern,
cloud-enabled, and Internet-connected applications with cross-platform
compatibility. It supports various applications, including web apps,
IoT, and mobile backends, allowing developers to use preferred tools
across Windows, macOS, and Linux. Offering seamless deployment to both
cloud and on-premises environments, ASP.NET Core operates on the .NET
Core runtime, empowering developers to create robust, scalable
applications. It represents a leaner and more modular evolution from
ASP.NET 4.0, unifying web UI and APIs, emphasizing testability, and
introducing innovations like Blazor for C\# usage in the browser.
ASP.NET Core is versatile, open-source, supports gRPC, and offers
flexible hosting options, showcasing its adaptability and
community-driven focus.

\subsection{Entity Framework}

Entity Framework is an open-source Object-Relational Mapping (ORM) 
framework by Microsoft for .NET applications, abstracting database
complexities through domain-specific classes. Operating at a higher
level of abstraction, it simplifies data-oriented application creation,
resulting in concise and productive code. Features include Entity Data
Model, LINQ queries, raw SQL execution, change tracking, transaction
management, built-in caching, concurrency control, and asynchronous
saving. It adheres to conventions-over-configuration programming and
provides migration commands for seamless database schema management,
making Entity Framework a robust tool for efficient database development
in .NET applications.


\subsection{MS SQL Server}

Microsoft SQL Server stands as a formidable relational database
management system (RDBMS) catering to a diverse range of applications in
corporate IT environments, including transaction processing, business
intelligence, and analytics. It holds a prominent position among the top
three database technologies, alongside Oracle Database and
IBM\textquotesingle s DB2. Operating on the standardized SQL programming
language, SQL Server empowers database administrators and IT
professionals to efficiently manage databases and query their data. The
system is intricately connected to Transact-SQL (T-SQL), a Microsoft
implementation of SQL that introduces proprietary programming
extensions, enhancing the functionality of the standard language. This
combination of robust features and integration with the widely-used SQL
language makes Microsoft SQL Server a key player in the database
technology landscape.


\subsection{Redis Cache}

Redis, an open-source in-memory data structure store, serves multiple
roles as a database, cache, message broker, and streaming engine. Its
versatility lies in supporting various data structures like strings,
hashes, lists, sets, sorted sets with range queries, bitmaps, hyperlogs,
geospatial indexes, and streams. Redis boasts built-in features such as
replication, Lua scripting, LRU (Least Recently Used) eviction,
transactions, and diverse levels of on-disk persistence. Additionally,
it ensures high availability through Redis Sentinel and automatic
partitioning with Redis Cluster. This makes Redis a robust and flexible
solution for handling real-time data storage, retrieval, and processing
in a wide range of applications.


\subsection{Elastic-search}

Elasticsearch is a key component in the Elastic Stack ecosystem, serving
as a distributed search and analytics engine. Alongside Logstash and
Beats, it collects, aggregates, and enriches data, while Kibana enables
interactive exploration and visualization. Elasticsearch excels in near
real-time search and analytics for diverse data types, offering
scalability and flexibility. Its applications range from adding search
functionality to apps and storing logs to employing machine learning and
processing genetic data for various research purposes.

\subsection{Azure Storage}

Azure Storage, Microsoft\textquotesingle s cloud storage platform,
offers highly available, massively scalable, and secure storage for
diverse data objects worldwide through HTTP or HTTPS. Client libraries
for popular languages, including .NET, Java, Python, JavaScript, C++,
and Go, facilitate application development. Azure Storage provides
user-friendly interfaces through the Azure portal and Storage Explorer.
It ensures data durability, high availability, and security through
redundancy, encryption, and access controls. With global accessibility,
scalability, and managed services, Azure Storage caters to modern
cloud-based storage needs, making it a robust and flexible solution for
various scenarios.


\subsection{Docker}

Docker, an open platform, revolutionizes software development by
decoupling applications from infrastructure and expediting delivery. It
aligns shipping, testing, and deployment methodologies, bridging the gap
between code creation and production. Docker\textquotesingle s
standardized environments in local containers ensure consistency,
particularly beneficial for CI/CD workflows. With highly portable
workloads, Docker supports dynamic workload management across various
environments, optimizing efficiency and scalability. Its fast,
lightweight design offers a cost-effective alternative to virtual
machines, making it ideal for high-density environments and
resource-efficient deployments.


\subsection{Nginx Load Balancer}

Nginx is a web server that can also function as a load balancer. Nginx
offers several load-balancing methods, including round-robin,
least-connected, and ip-hash. The round-robin method distributes
requests to the application servers in a circular fashion, while the
least-connected method assigns the next request to the server with the
least number of active connections. In this context of the project, we
will use the round-robin load-balancing method.