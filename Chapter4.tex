\chapter{TECHNOLOGIES}

\section{Front-end}


In this section, we will dive into Next.js and Tailwind CSS. Next.js is
a React framework with features like server-side rendering, while
Tailwind CSS is a utility-first framework for streamlined styling
through pre-designed classes. These tools play a key role in creating
efficient and scalable front-end experiences.


\subsection{NextJS}

Next.js, a robust React framework, tackles challenges in traditional
client-side rendering. It utilizes server-side rendering and static site
generation to address accessibility, security, page loading times, and
Search engine optimization (SEO) concerns. Popular in the React
ecosystem, Next.js optimizes the user experience by pre-rendering pages
on the server. Advantages include enhanced performance with server-side
rendering, seamless React integration for dynamic interfaces, static
site generation for optimal website performance, intuitive file
system-based routing, automatic code splitting, flexible styling
approaches, and compatibility with TypeScript for improved code quality.

In our application, we use Next.js to create a dynamic and responsive
front-end experience, leveraging its server-side rendering capabilities
to enhance performance and SEO.

\subsection{Tailwind CSS}

Tailwind CSS, a utility-first framework, transforms web styling by
providing an extensive set of classes directly within HTML. This
accelerates development by offering granular control over layout, color,
spacing, and more. Unlike traditional CSS, Tailwind\textquotesingle s
unique approach eliminates the need for external files, enabling
seamless customization and efficient creation of visually appealing and
responsive interfaces. Its utility-first approach minimizes unused CSS,
optimizing performance, and its low learning curve, active community,
and integration flexibility make it an efficient choice for building
well-styled and maintainable web applications.

With Tailwind CSS, we can easily design and style our application,
ensuring consistency and responsiveness across different devices and
screen sizes.

\subsection{Retool}
Retool is a rapid development platform for building internal software 
quickly and efficiently. It features a drag-and-drop interface for creating 
custom applications, supports integration with a wide range of databases and 
APIs, and allows for the incorporation of custom JavaScript. With Retool, 
users can easily combine data from multiple sources and implement complex 
functionalities without extensive coding. Its built-in permissions and access 
controls ensure data security, making Retool a valuable tool for enhancing 
internal workflows and productivity.

We use Retool to create dashboards and internal tools that facilitate
the management and monitoring of our application, enabling efficient
data visualization and analysis for administrators.

\section{Back-end}

In this section, we will explore the ASP.NET Core platform, Entity
Framework, MS SQL Server, Redis Cache, ML.NET, and Elastic-search. These
technologies play a crucial role in developing robust back-end
functionalities, ensuring efficient data management, and enhancing
scalability and performance in web applications.

\subsection{ASP.NET core platform}

ASP.NET Core, an open-source framework, excels in developing modern,
cloud-enabled, and Internet-connected applications with cross-platform
compatibility. It supports various applications, including web apps,
IoT, and mobile backends, allowing developers to use preferred tools
across Windows, macOS, and Linux. Offering seamless deployment to both
cloud and on-premises environments, ASP.NET Core operates on the .NET
Core runtime, empowering developers to create robust, scalable
applications. It represents a leaner and more modular evolution from
ASP.NET 4.0, unifying web UI and APIs, emphasizing testability, and
introducing innovations like Blazor for C\# usage in the browser.
ASP.NET Core is versatile, open-source, supports gRPC, and offers
flexible hosting options, showcasing its adaptability and
community-driven focus.

In our application, we leverage ASP.NET Core to build a reliable and
scalable back-end infrastructure, ensuring optimal performance and
security.

\subsection{Entity Framework}

Entity Framework is an open-source Object-Relational Mapping (ORM)
framework by Microsoft for .NET applications, abstracting database
complexities through domain-specific classes. Operating at a higher
level of abstraction, it simplifies data-oriented application creation,
resulting in concise and productive code. Features include Entity Data
Model, LINQ queries, raw SQL execution, change tracking, transaction
management, built-in caching, concurrency control, and asynchronous
saving. It adheres to conventions-over-configuration programming and
provides migration commands for seamless database schema management,
making Entity Framework a robust tool for efficient database development
in .NET applications.

We utilize Entity Framework to streamline database interactions and
management, ensuring data integrity and consistency in our application.


\subsection{MS SQL Server}

Microsoft SQL Server stands as a formidable relational database
management system (RDBMS) catering to a diverse range of applications in
corporate IT environments, including transaction processing, business
intelligence, and analytics. It holds a prominent position among the top
three database technologies, alongside Oracle Database and
IBM\textquotesingle s DB2. Operating on the standardized SQL programming
language, SQL Server empowers database administrators and IT
professionals to efficiently manage databases and query their data. The
system is intricately connected to Transact-SQL (T-SQL), a Microsoft
implementation of SQL that introduces proprietary programming
extensions, enhancing the functionality of the standard language. This
combination of robust features and integration with the widely-used SQL
language makes Microsoft SQL Server a key player in the database
technology landscape.

We leverage Microsoft SQL Server to store and manage data in our
application, ensuring reliability, scalability, and performance in
database operations.


\subsection{Redis Cache}

Redis, an open-source in-memory data structure store, serves multiple
roles as a database, cache, message broker, and streaming engine. Its
versatility lies in supporting various data structures like strings,
hashes, lists, sets, sorted sets with range queries, bitmaps, hyperlogs,
geospatial indexes, and streams. Redis boasts built-in features such as
replication, Lua scripting, LRU (Least Recently Used) eviction,
transactions, and diverse levels of on-disk persistence. Additionally,
it ensures high availability through Redis Sentinel and automatic
partitioning with Redis Cluster. This makes Redis a robust and flexible
solution for handling real-time data storage, retrieval, and processing
in a wide range of applications.

We utilize Redis Cache to enhance performance and scalability in our
application, leveraging its in-memory data storage capabilities for
efficient caching and data processing.


\subsection{Elastic-search}

Elasticsearch is a key component in the Elastic Stack ecosystem, serving
as a distributed search and analytics engine. Alongside Logstash and
Beats, it collects, aggregates, and enriches data, while Kibana enables
interactive exploration and visualization. Elasticsearch excels in near
real-time search and analytics for diverse data types, offering
scalability and flexibility. Its applications range from adding search
functionality to apps and storing logs to employing machine learning and
processing genetic data for various research purposes.

We use Elasticsearch to enhance search functionality in our application,
providing users with efficient and accurate search results across
different data sources.

\subsection{ML.NET}
ML.NET is an open-source, cross-platform machine learning framework for .NET 
developers that enables integration of custom machine learning models into .NET 
applications. It encompasses an API, which consists of different NuGet packages, 
a Visual Studio extension called Model Builder, and a command-line interface that's 
installed as a .NET tool.

With ML.NET, we train and deploy machine learning models in our application,
to enhance user experience and provide intelligent breeds classification
based on images.

\subsection{Docker}
Docker is a platform for developing, shipping, and running applications in containers.
It provides a consistent environment for applications to run in, ensuring that they
work the same way across different environments. Docker containers are lightweight,
portable, and isolated, making them ideal for deploying applications in various
environments, from development to production.

We use Docker to containerize our application components, set up development
environments, and streamline deployment processes, ensuring consistency and
reliability in our application deployment.

\section{Deployment}

\subsection{ImgBB}
ImgBB is a user-friendly image hosting and sharing service that allows users 
to upload and share images easily. It supports various image formats such as 
JPEG, PNG, and GIF, making it versatile for different types of visual content. 
ImgBB provides features like direct image links, BBCode, and HTML thumbnails, 
which are particularly useful for embedding images in forums, websites, and 
social media. The service also includes functionalities like image resizing and 
expiration settings, giving users control over how and for how long their images 
are displayed.

We use ImgBB to host in our application, enabling seamless
image uploading for users.

\subsection{Vercel}
Vercel is a cloud platform for static sites and serverless functions that enables
developers to deploy web applications with ease. It supports various frameworks
like Next.js, Gatsby, and React, providing seamless integration and deployment
options. Vercel offers features like automatic deployments, preview URLs, and
custom domains, making it convenient for developers to showcase their projects
and collaborate with team members. The platform also includes performance
analytics, serverless functions, and environment variables, enhancing the
development and deployment experience for modern web applications.

We use Vercel to deploy our application, leveraging its features for
efficient hosting and deployment of Next.js projects.

\subsection{Azure Virtual Machine}
Azure Virtual Machines (VMs) are scalable computing resources provided by
Microsoft Azure for running applications and workloads in the cloud. VMs offer
flexibility in terms of operating systems, configurations, and performance
requirements, allowing users to customize their computing environment based on
specific needs. Azure VMs support various use cases, including development and
testing, production workloads, and high-performance computing. With features like
autoscaling, load balancing, and virtual networking, Azure VMs provide a
comprehensive cloud computing solution for deploying and managing applications
in the cloud.

We use Azure Virtual Machines to host our application, leveraging the
scalability and flexibility of Azure cloud services for efficient
deployment and management of our web application.