\chapter{SYSTEM TESTING}

With the completion of system implementation, our focus now shifts towards ensuring the robustness and functionality of the system through testing. This section presents the systematic approach taken for system testing, including end-to-end testing and API testing, to validate the functionality and performance of the implemented system

\section{API testing}
This section delves into testing the website's API for functionality using 
Postman. Postman's versatile features were utilized to conduct thorough 
assessments of the API endpoints, ensuring data integrity and performance 
across diverse scenarios. Through rigorous experimentation and analysis, 
Postman facilitated the establishment of a robust validation framework, 
affirming the reliability of the website's API under different conditions.

\subsection{API endpoints}
The website's API endpoints were meticulously tested to ensure their
functionality and performance across various scenarios. The API endpoints
were designed to handle different types of requests, including GET, POST,
PUT, and DELETE, to interact with the database and retrieve or modify
data. By testing these endpoints, we could verify the accuracy and
reliability of the API, ensuring that it could handle different types of
requests and return the expected results.

\subsection{Automation API Testing with Postman}
Postman was used to test the website's API endpoints, leveraging its
versatile features to conduct comprehensive assessments. Postman's
user-friendly interface and powerful testing capabilities enabled us to
create test cases, execute requests, and analyze responses efficiently.
By defining test scripts and assertions, we could validate the API
endpoints' functionality and performance, ensuring that they met the
specified requirements.
In this section, we focus on the response data, status code, and response
time of the API endpoints, evaluating their accuracy and reliability.
For status, we have this code on Postman
\begin{lstlisting}[caption=Code to test status code, label={lst:status-code}]
    pm.test("Status code is 200", function () {
        pm.response.to.have.status(200);
    });
\end{lstlisting}
To test the status code, we used the following code in Postman:
\begin{lstlisting}[caption=Code to test response time, label={lst:response-time}]
    pm.test("Response time is less than 200ms", function () {
        pm.expect(pm.response.responseTime).to.be.below(200);
    });
\end{lstlisting}
To test the response data, we used this code in Postman:
\begin{lstlisting}[caption=Code to test response data, label={lst:response-data}]
    pm.test("Response contains errorCode: 10006 and errorMessage: Wrong login type", function () {
    var responseData = pm.response.json();
    pm.expect(responseData).to.eql({ errorCode: 10006, errorMessage: "Wrong login type" });
});
\end{lstlisting}

\subsection{API testing scenarios}
Various testing scenarios were explored to evaluate the website's API
endpoints thoroughly. These scenarios included testing different types of
requests, handling error conditions, and assessing the API's performance
under load. By simulating real-world usage scenarios and edge cases, we
could identify potential issues and validate the API's behavior in
different situations. This comprehensive testing approach helped ensure
the reliability and robustness of the website's API, enhancing its overall
quality and performance. 

Here is all the test cases for the API endpoints. All the API endpoints
here have average response time of 734ms. The test cases are divided into:


\begin{longtblr}[
    caption = {API Testing for Adoption Form},
    label = {tblr:api_adoption_form},
  ]{
    hlines, vlines,
    colspec = {X[0.5,l] X[3,l] X[4,l] X[5,l] X[1.5,l] X[2,l]},
    row{1} = {font=\bfseries},
  }
  Id                & Api                                          & Test-case               & Expected result                                                             & Result & \SetCell[c=1]{c}Response time \\
  \SetCell[r=3]{c}1 & \SetCell[r=3]{c}POST - /AdoptionForm         & 1. Unauthorized        & Error code 400 with message "Unauthorized".                                 & passed   & \SetCell[r=3]{c}872ms         \\*
                    &                                              & 2. Empty body          & Error code 401 with corresponding empty fields.                             & passed   &                               \\
                    &                                              & 3. Correct body        & Status 200.                                                                 & passed                                   \\
  \SetCell[r=3]{c}2 & \SetCell[r=3]{c}GET - /AdoptionForm/ Precheck & 1. Is adopted          & Error code 12003 with message "You have sent request for this pet already". & passed   & \SetCell[r=3]{c}812ms        \\*
                    &                                              & 2. Not adopted         & Success with status 200                                                     & passed   &                               \\
                    &                                              & 3. Adopt from yourself & Error code 12002 with message "Can not adopt your own pet"                  & passed   &                               \\
  \SetCell[r=2]{c}3 & \SetCell[r=2]{c}GET - /AdoptionForm/Incoming & 1. Unauthorized        & Error code 400 with message "Unauthorized".                                 & passed   & \SetCell[r=2]{c}527ms         \\*
                    &                                              & 2. Authorized          & Success with status 200                                                     & passed   &                               \\
  \SetCell[r=2]{c}4 & \SetCell[r=2]{c}GET - /AdoptionForm/Sent     & 1. Unauthorized        & Error code 400 with message "Unauthorized".                                 & passed   & \SetCell[r=2]{c}644ms         \\*
                    &                                              & 2. Authorized          & Success with status 200                                                     & passed   &                               \\
  \SetCell[r=2]{c}5 & \SetCell[r=2]{c}GET - /CountUnreadIncoming   & 1. Unauthorized        & Error code 400 with message "Unauthorized".                                 & passed   & \SetCell[r=2]{c}613ms         \\*
                    &                                              & 2. Authorized          & Success with status 200                                                     & passed   &                               \\
\end{longtblr}

\begin{longtblr}[
        caption = {API Testing for Authentication Function},
        label = {tblr:api_Authentication},
    ]{
        hlines, vlines,
        colspec = {X[0.5,l] X[3,l] X[4,l] X[5,l] X[1.5,l] X[2,l]},
        row{1} = {font=\bfseries},
    }
    Id                & Api                                             & Test-case             & Expected result                                       & Result & \SetCell[c=1]{c}Response time \\
    \SetCell[r=5]{c}1 & \SetCell[r=5]{c}POST - /Authentication/Login    & 1. Wrong email       & Error code 10001 with message "Invalid credential".   & passed   & \SetCell[r=5]{c}669ms         \\*
                      &                                                 & 2. Wrong password    & Error code 10001 with message "Invalid credential".   & passed   &                               \\
                      &                                                 & 3. Correct account   & Status 200.                                           & passed   &                               \\
                      &                                                 & 4. Wrong login type  & Error code 10006 with message "Wrong login type".     & passed   &                               \\
                      &                                                 & 5. Empty body        & Error code 400 with corresponding empty fields        & passed   &                               \\
    \SetCell[r=2]{c}2 & \SetCell[r=2]{c} GET - /Authentication/Logout   & 1. Unauthorized      & Error code 401 with message "Unauthorized".           & passed   & \SetCell[r=2]{c}838ms         \\*
                      &                                                 & 2. Authorized        & Status 200.                                           & passed   &                               \\
    \SetCell[r=2]{c}3 & \SetCell[r=2]{c}GET - /Authentication/Refresh   & 1. Unauthorized      & Error code 401 with message "Invalid security token". & passed   & \SetCell[r=2]{c}750ms         \\*
                      &                                                 & 2. Authorized        & Status 200.                                           & passed   &                               \\
\end{longtblr}

\begin{longtblr}[
    caption = {API Testing for Blog Function},
    label = {tblr:api_Blog},
  ]{
    hlines, vlines,
    colspec = {X[0.5,l] X[3,l] X[4,l] X[5,l] X[1.5,l] X[2,l]},
    row{1} = {font=\bfseries},
  }
  Id                & Api                                         & Test-case                   & Expected result                                 & Result & Response time         \\
  \SetCell[r=4]{c}1 & \SetCell[r=4]{c}POST - /Blog                & 1. Unauthorized            & Error code 401 with message "Unauthorized".     & passed   & \SetCell[r=4]{c}693ms \\*
                    &                                             & 2. Wrong role              & Error code 400 with message "Forbidden".        & passed   &                       \\
                    &                                             & 3. Correct body            & Status 200.                                     & passed   &                       \\
                    &                                             & 4. Empty body              & Error code 400 with corresponding empty fields. & passed   &                       \\
  \SetCell[r=4]{c}2 & \SetCell[r=4]{c}PUT - /Blog                 & 1. Unauthorized            & Error code 401 with message "Unauthorized".     & passed   & \SetCell[r=4]{c}700ms \\*
                    &                                             & 2. Wrong role              & Error code 400 with message "Forbidden".        & passed   &                       \\
                    &                                             & 3. Correct body            & Status 200.                                     & passed   &                       \\
                    &                                             & 4. Empty body              & Error code 400 with corresponding empty fields. & passed   &                       \\
  \SetCell[r=2]{c}3 & \SetCell[r=2]{c}POST - /Blog/Get            & 1. Empty body              & Response with status 500.                       & passed   & \SetCell[r=2]{c}681ms \\*
                    &                                             & 2. Correct body            & Status 200.                                     & passed   &                       \\
  \SetCell[r=4]{c}4 & \SetCell[r=4]{c}DELETE - /Blog/\{blog\_id\} & 1. Unauthorized            & Error code 401 with message "Unauthorized".     & passed   & \SetCell[r=4]{c}677ms \\*
                    &                                             & 2. Delete existed blog     & Success with status 200.                        & passed   &                       \\
                    &                                             & 3. Delete non-existed blog & Error code 400 with message "Forbidden".        & passed   &                       \\
                    &                                             & 4. Delete empty blog\_id   & Error with status 405.                          & passed   &                       \\
  \SetCell[r=3]{c}5 & \SetCell[r=3]{c}POST - /Blog/User           & 1. Unauthorized            & Error code 401 with message "Unauthorized".     & passed   & \SetCell[r=3]{c}667ms \\*
                    &                                             & 2. Empty body              & Response with status 500.                       & passed   &                       \\
                    &                                             & 3. Correct body            & Status 200.                                     & passed   &                       \\
\end{longtblr}


\begin{longtblr}[
    caption = {API Testing for Comment Function},
    label = {tblr:api_comment},
  ]{
    hlines, vlines,
    colspec = {X[0.6,l] X[2.3,l] X[3.3,l] X[4.5,l] X[0.8,l] X[1.7,l]},
    row{1} = {font=\bfseries},
  }
    Id & Api & Testcase & Expected result & Result & \SetCell[c=1]{c}Response time \\
    \SetCell[r=3]{c}1 & \SetCell[r=3]{c}GET - /Comment/blog/\{blog\_id\} & 1. Correct id & Status 200 & Pass & \SetCell[r=3]{c}708ms \\
    & & 2. Wrong blog id & Status 400 with message "The value \{blog\_id\} id not valid" & Pass & \\
    & & 3. Empty blog id & Status 405 & Pass & \\
    \SetCell[r=3]{c}2 & \SetCell[r=3]{c}GET - /Comment/post/\{blog\_id\} & 1. Correct id & Status 200 & Pass & \SetCell[r=3]{c}629ms \\
    & & 2. Wrong blog id & Status 400 with message "The value \{blog\_id\} id not valid" & Pass & \\
    & & 3. Empty blog id & Status 405 & Pass & \\
    \SetCell[r=3]{c}3 & \SetCell[r=3]{c}POST - /Comment & 1. Unauthorized & Error code 401 with message "Unauthorized". & Pass & \SetCell[r=3]{c}779ms \\
    & & 2. Comment with text & Status 200. & Pass & \\
    & & 3. Empty comment & Error code 401 with message "Can not be empty". & Pass & \\
    \SetCell[r=3]{c}4 & \SetCell[r=3]{c}DEL - /Comment/\{comment\_id\} & 1. Unauthorized & Error code 401 with message "Unauthorized". & Pass & \SetCell[r=3]{c}697ms \\
    & & 2. Delete self comment & Status 200. & Pass & \\
    & & 3. Delete others comment & Error code 401 with message "Unauthorized". & Pass & \\
  \end{longtblr}

\begin{longtblr}[
    caption = {API Testing for Location Function},
    label = {tblr:api_location},
  ]{
    hlines, vlines,
    colspec = {X[0.5,l] X[3,l] X[4,l] X[5,l] X[1.5,l] X[2,l]},
    row{1} = {font=\bfseries},
  }
    Id & Api & Testcase & Expected result & Result & \SetCell[c=1]{c}Response time \\
    \SetCell[r=8]{c}1 & \SetCell[r=8]{c}GET - /Location & 1. Unauthorized & Error code 401 with message "Unauthorized". & Pass & \SetCell[r=8]{c}707ms \\*
    & & 2. Empty level and code & Status 200 with empty data & Pass & \\
    & & 3. Level = 1 & Return all provinces & Pass & \\
    & & 4. Level = 2 & Status 200 with empty data & Pass & \\
    & & 5. Code = 1 & Status 200 with empty data & Pass & \\
    & & 6. Level = 2, Code = 79 & Return all districts corresponding to level & Pass & \\
    & & 7. Level = 3, Code = 775 & Return all wards corresponding to level & Pass & \\
    & & 8. Level = wrong, Code = wrong & Status 400 with message "The value 'wrong' is not valid for level" & Pass & \\
  \end{longtblr}

\begin{longtblr}[
    caption = {API Testing for Notification Function},
    label = {tblr:api_notification},
  ]{
    hlines, vlines,
    colspec = {X[0.5,l] X[3,l] X[4,l] X[5,l] X[1.5,l] X[2,l]},
    row{1} = {font=\bfseries},
  }
  Id                & Api                                                       & Test-case                 & Expected result                             & Result & \SetCell[c=1]{c}Response time \\
  \SetCell[r=2]{c}1 & \SetCell[r=2]{c} GET - /Notification                      & 1. Unauthorized          & Error code 401 with message "Unauthorized". & passed   & \SetCell[r=2]{c}651ms         \\*
                    &                                                           & 2. Authorize             & Status 200 with message "true"              & passed   &                               \\
  \SetCell[r=2]{c}2 & \SetCell[r=2]{c}DEL - /Notification                       & 1. Unauthorized          & Error code 401 with message "Unauthorized". & passed   & \SetCell[r=2]{c}678ms         \\*
                    &                                                           & 2. Authorize             & Status 200 with message "true"              & passed   &                               \\
  \SetCell[r=3]{c}3 & \SetCell[r=3]{c}PUT - /Notification/ \{notification\_id\} & 1. Unauthorized          & Error code 401 with message "Unauthorized". & passed   & \SetCell[r=3]{c}685ms         \\*
                    &                                                           & 2. Update others comment & Status 200 with message "false"             & passed   &                               \\
                    &                                                           & 3. Update self comment   & Status 200 with message "true"              & passed   &                               \\
\end{longtblr}

\begin{longtblr}[
    caption = {API Testing for Payment Function},
    label = {tblr:api_payment},
  ]{
    hlines, vlines,
    colspec = {X[0.6,l] X[2.3,l] X[3.3,l] X[4.5,l] X[0.8,l] X[1.7,l]},
    row{1} = {font=\bfseries},
  }
    Id & Api & Testcase & Expected result & Result & \SetCell[c=1]{c}Response time \\
    \SetCell[r=3]{c}1 & \SetCell[r=3]{c}GET - /Payment/Token & 1. Unauthorized & Error code 401 with message "Unauthorized". & Pass & \SetCell[r=3]{c}617ms \\
    & & 2. Wrong role & Status 400 with message "Forbidden" & Pass & \\
    & & 3. Correct role & Status 200 & Pass & \\
    \SetCell[r=2]{c}2 & \SetCell[r=2]{c}POST - /Payment & 1. Unauthorized & Error code 401 with message "Unauthorized". & Pass & \SetCell[r=2]{c}648ms \\
    & & 2. Wrong role & Status 400 with message "Forbidden" & Pass & \\
    \SetCell[r=3]{c}3 & \SetCell[r=3]{c}GET - /Payment/AdvertisementType & 1. Unauthorized & Error code 401 with message "Unauthorized". & Pass & \SetCell[r=3]{c}666ms \\
    & & 2. Wrong role & Status 400 with message "Forbidden" & Pass & \\
    & & 3. Correct role & Status 200 & Pass & \\
  \end{longtblr} 

\begin{longtblr}[
    caption = {API Testing for Pet Function},
    label = {tblr:api_pet},
  ]{
    hlines, vlines,
    colspec = {X[0.5,l] X[3,l] X[4,l] X[5,l] X[1.5,l] X[2,l]},
    row{1} = {font=\bfseries},
  }
    Id & Api & Test-case & Expected result & Result & \SetCell[c=1]{c}Response time \\
    \SetCell[r=3]{c}1 & \SetCell[r=3]{c}POST - /Pet & 1. Unauthorized & Error code 401 with message "Unauthorized". & passed & \SetCell[r=3]{c}620ms \\*
    & & 3. Correct body & Status 200. & passed & \\
    & & 4. Empty body & Error code 400 with corresponding empty fields & passed & \\
    \SetCell[r=4]{c}2 & \SetCell[r=4]{c}PUT - /Pet & 1. Unauthorized & Error code 401 with message "Unauthorized". & passed & \SetCell[r=4]{c}723ms \\*
    & & 2. Wrong field value & Error status 400 & passed & \\
    & & 3. Correct body & Status 200. & passed & \\
    & & 4. Empty body & Error code 400 with corresponding empty fields & passed & \\
    \SetCell[r=9]{c}3 & \SetCell[r=9]{c}POST - /Pet/Get & 1. Unauthorized & Error code 401 with message "Unauthorized". & passed & \SetCell[r=9]{c}757ms \\*
    & & 2. Empty body & Error code 400 with corresponding empty fields & passed & \\
    & & 3. Test filter species & Response contains pet with correct species & passed & \\
    & & 4. Test filter sex & Response contains pet with correct sex & passed & \\
    & & 5. Test filter age & Response contains pet with correct age & passed & \\
    & & 6. Test filter color & Response contains pet with correct color & passed & \\
    & & 7. Test filter size & Response contains pet with correct size & passed & \\
    & & 8. Test filter isSterilized & Response contains pet with correct isSterilized & passed & \\
    & & 9. Test filter Vaccinated & Response contains pet with correct Vaccinated & passed & \\
    \SetCell[r=3]{c}4 & \SetCell[r=3]{c}DELETE - /Pet/\{pet\_id\} & 1. Unauthorized & Error code 401 with message "Unauthorized". & passed & \SetCell[r=3]{c}719ms \\*
    & & 2. Delete existing pet & Success with status 200 & passed & \\
    & & 3. Delete non-existing pet & Error code 11001 with message "The pet is not found". & passed & \\
  \end{longtblr}


\begin{longtblr}[
    caption = {API Testing for Report Function},
    label = {tblr:api_report},
  ]{
    hlines, vlines,
    colspec = {X[0.6,l] X[2.5,l] X[3,l] X[4,l] X[0.8,l] X[1.5,l]},
    row{1} = {font=\bfseries},
  }
    Id & Api & Testcase & Expected result & Result & \SetCell[c=1]{c}Response time \\
    \SetCell[r=8]{c}1 & \SetCell[r=8]{c}{POST - /Report/PreReport} & 1. Unauthorized & Error code 401 with message "Unauthorized". & Pass & \SetCell[r=8]{c}679ms \\
    & & 2. Report yourself & Status 200 with message "false" & Pass & \\
    & & 3. Report other user & Status 200 with message "true" & Pass & \\
    & & 4. Report other user with wrong entity & Status 200 with message "false" & Pass & \\
    & & 5. Report your blog & Status 200 with message "false" & Pass & \\
    & & 6. Report others' blog & Status 200 with message "true" & Pass & \\
    & & 7. Report your pet & Status 200 with message "false" & Pass & \\
    & & 8. Report others' pet & Status 200 with message "true" & Pass & \\
    \SetCell[r=8]{c}2 & \SetCell[r=8]{c}{POST - /Report/Report} & 1. Unauthorized & Error code 401 with message "Unauthorized". & Pass & \SetCell[r=8]{c}733ms \\
    & & 2. Report yourself & Status 200 with message "false" & Pass & \\
    & & 3. Report other user & Status 200 with message "true" & Pass & \\
    & & 4. Report other user with wrong entity & Status 200 with message "false" & Pass & \\
    & & 5. Report your blog & Status 200 with message "true" & Pass & \\
    & & 6. Report others' blog & Status 200 with message "true" & Pass & \\
    & & 7. Report your pet & Status 200 with message "false" & Pass & \\
    & & 8. Report others' pet & Status 200 with message "true" & Pass & \\
  \end{longtblr}

\begin{longtblr}[
    caption = {API Testing for User Function},
    label = {tblr:api_user},
  ]{
    hlines, vlines,
    colspec = {X[0.6,l] X[2.5,l] X[3,l] X[4,l] X[0.8,l] X[1.5,l]},
    row{1} = {font=\bfseries},
  }
    Id & Api & Testcase & Expected result & Result & \SetCell[c=1]{c}Response time \\
    \SetCell[r=2]{c}1 & \SetCell[r=2]{c}GET - /User/CurrentUser & 1. Unauthorized & Error code 401 with message "Unauthorized". & Pass & \SetCell[r=2]{c}694ms \\
    & & 2. Authorized & Status 200. & Pass & \\
    \SetCell[r=2]{c}2 & \SetCell[r=2]{c}GET - /User/CurrentUserCore & 1. Unauthorized & Error code 401 with message "Unauthorized". & Pass & \SetCell[r=2]{c}884ms \\
    & & 2. Authorized & Status 200. & Pass & \\
    \SetCell[r=2]{c}3 & \SetCell[r=2]{c}GET - /User/OtherUser & 1. Unauthorized & Status 200. & Pass & \SetCell[r=2]{c}664ms \\
    & & 2. Authorized & Status 200. & Pass & \\
    \SetCell[r=3]{c}4 & \SetCell[r=3]{c}POST - /User/ForgotPassword & 1. Wrong login type & Error code 10006 with message "Wrong login type". & Pass & \SetCell[r=3]{c} \\
    & & 2. Wrong email & Error code 10000 with message "Provided email is not correct" & Pass & \\
    & & 3. Correct email & Status 200. & Pass & \\
    \SetCell[r=3]{c}5 & \SetCell[r=3]{c}PUT - /User & 1. Unauthorized & Error code 401 with message "Unauthorized". & Pass & \SetCell[r=3]{c}654ms \\
    & & 2. Empty body & Error code 400 with corresponding empty fields & Pass & \\
    & & 3. Correct body & Status 200 & Pass & \\
    \SetCell[r=3]{c}6 & \SetCell[r=3]{c}PUT - /User/UpdateAvatar & 1. Unauthorized & Error code 401 with message "Unauthorized". & Pass & \SetCell[r=3]{c}670ms \\
    & & 2. Correct body & Status 200 & Pass & \\
    & & 3. Empty image & Error code 400 with message "Image cannot be empty" & Pass & \\
    \SetCell[r=3]{c}7 & \SetCell[r=3]{c}POST - /User/UpgradeAccount & 1. Unauthorized & Error code 401 with message "Unauthorized". & Pass & \SetCell[r=3]{c}593ms \\
    & & 2. Has not applied & Status 200 & Pass & \\
    & & 3. Has applied & Status 200 with message False & Pass & \\
  \end{longtblr}

\section{End-to-End Testing}

End-to-end testing is a software testing method that validates the system as a whole, ensuring that all components work together as expected. This type of testing is essential to ensure that the system meets the requirements and performs as intended. In this section, we present the end-to-end testing process for the implemented system.

\subsection{Automation End-to-End Testing with Cypress}

To automate the end-to-end testing process, we choose Cypress, which is a modern JavaScript-based end-to-end testing framework designed for web applications. It provides a fast, reliable, and easy-to-use testing solution for developers and QA engineers. Cypress comes with a rich set of features such as real-time reloading, automatic waiting, and built-in assertions,  allowing for efficient and effective testing of web applications.

\begin{figure}[H]
    \centering
    \includegraphics[width=0.7\textwidth]{Figures/cypress_script.png}
    \caption{Automation Script}
    \label{fig:cypress-script}
\end{figure}

\begin{figure}[H]
    \centering
    \includegraphics[width=0.7\textwidth]{Figures/cypress_auto.png}
    \caption{Automation tests done by Cypress}
    \label{fig:cypress-test}
\end{figure}

\subsection{Test Cases}
With Cypress, we tested our system features to make sure the system ran as expected and satisfied requirements. Below are some test cases of the key features and test data for them, these test cases were tested on multiple browser and environments.

\subsubsection*{Test data}

Normal User account:

\quad Email: vejox62251@facais.com - Password: 123456789

Organization User account:

\quad Email:  jayoki3306@facais.com - Password: 123456789

Owned-pet-id: baf3745f-d0d7-40c7-8084-5b55cb2f2ce3

\subsubsection*{Authentication}

\begin{longtblr}[
    caption = {Non-restricted Page Accessibility Test},
    label = {tblr:non_restricted_page_accessibility},
  ]{
    vline{1-3} = {-}{},
    hline{-} = {1-2}{},
    colspec={X[2,l] X[5, l]},
  }
  \textbf{Description} & \textbf{Non-restricted page accessibility} \\
  \textbf{Purpose} & {
    Test whether the guest user can access the pages that do not require authentication.
  } \\
  \textbf{Precondition} & {

  } \\
  \textbf{Steps} & {
    1. Click the “Nhận nuôi” button on the Nav bar.
    \\2. System shows the page content.
    \\3. Click the first pet card on the pet list.
    \\4. System shows the page content.
    \\5. Click the “Tin tức” button on the Nav bar.
    \\6. System shows the page content.
    \\7. Click the first blog card on the blog list.
    \\8. System shows the page content.
  } \\
  \textbf{Expected result} & {
    System does not direct the user to the login page.
  } \\
  \textbf{Execution results} & {
    PASSED
  } \\
\end{longtblr}




\begin{longtblr}[
    caption = {Restricted Page Accessibility Test},
    label = {tblr:restricted_page_accessibility},
  ]{
    vline{1-3} = {-}{},
    hline{-} = {1-2}{},
    colspec={X[2,l] X[5,l]},
  }
  \textbf{Description} & \textbf{Restricted page accessibility} \\
  \textbf{Purpose} & {
    Test whether the guest user can access the pages that do require authentication.
  } \\
  \textbf{Pre-condition} & {

  } \\
  \textbf{Steps} & {
    1. Click the “Cho thú cưng” button on the Nav bar.
  } \\
  \textbf{Expected result} & {
    System directs the user to the login page.
  } \\
  \textbf{Execution results} & {
    PASSED
  } \\
\end{longtblr}


\begin{longtblr}[
    caption = {Normal Account Accessibility Test},
    label = {tblr:normal_account_accessibility},
  ]{
    vline{1-3} = {-}{},
    hline{-} = {1-2}{},
    colspec={X[2,l] X[5,l]},
  }
  \textbf{Description} & \textbf{Normal account accessibility} \\
  \textbf{Purpose} & {
    Test whether the authenticated user with a normal account can access the pages that require authentication at the normal account level.
  } \\
  \textbf{Precondition} & {
    User logged in using a normal account.
  } \\
  \textbf{Steps} & {
    1. Click the “Cho thú cưng” button on the Nav bar.
  } \\
  \textbf{Expected result} & {
    System shows the restricted page content to the authenticated user.
  } \\
  \textbf{Execution results} & {
    PASSED
  } \\
\end{longtblr}


\begin{longtblr}[
    caption = {Organization Account Accessibility Test},
    label = {tblr:organization_account_accessibility},
  ]{
    vline{1-3} = {-}{},
    hline{-} = {1-2}{},
    colspec={X[2,l] X[5,l]},
  }
  \textbf{Description} & \textbf{Organization account accessibility} \\
  \textbf{Purpose} & {
    Test whether the authenticated user with an organization account can access the pages that require authentication at the organization account level.
  } \\
  \textbf{Precondition} & {
    User logged in using an organization account.
  } \\
  \textbf{Steps} & {
    1. Click the User avatar at top right.
    \\2. Click the User profile option.
    \\3. System directs to User profile page.
  } \\
  \textbf{Expected result} & {
    System shows the profile of the organization with the create blog option.
  } \\
  \textbf{Execution results} & {
    PASSED
  } \\
\end{longtblr}



\subsubsection*{Filter pet}

\begin{longtblr}[
    caption = {Filter Pets by Species Test},
    label = {tblr:filter_pets_by_species},
  ]{
    vline{1-3} = {-}{},
    hline{-} = {1-2}{},
    colspec={X[2,l] X[5,l]},
  }
  \textbf{Description} & \textbf{Filter pets by species} \\
  \textbf{Purpose} & {
    Test whether the pet list will be filtered by species.
  } \\
  \textbf{Precondition} & {
    User at search pet page.
  } \\
  \textbf{Steps} & {
    1. Click “Loài” button at the filter bar.
    \\2. Click option “Chó”.
  } \\
  \textbf{Expected result} & {
    System shows the list of dogs.
  } \\
  \textbf{Execution results} & {
    PASSED
  } \\
\end{longtblr}


The same test was done with other options: breed, age, color, sex, size, vaccinated, and spayed; which also yielded the same results. But there is a special case where breeds should be displayed according to the chosen species, so it is a separate case.

\begin{longtblr}[
    caption = {Breed Options Display According to Species Test},
    label = {tblr:breed_options_display},
  ]{
    vline{1-3} = {-}{},
    hline{-} = {1-2}{},
    colspec={X[2,l] X[5,l]},
  }
  \textbf{Description} & \textbf{Breed options display according to species} \\
  \textbf{Purpose} & {
    Test whether the breed options are displayed according to the chosen species.
  } \\
  \textbf{Precondition} & {
    User at search pet page.
  } \\
  \textbf{Steps} & {
    1. Click “Loài” button at the filter bar.
    \\2. Click option “Chó”.
    \\3. Click “Giống” button at the filter bar.
  } \\
  \textbf{Expected result} & {
    System shows a list of breeds of dogs available on the website.
  } \\
  \textbf{Execution results} & {
    PASSED
  } \\
\end{longtblr}


\subsubsection*{Account Configuration}

\begin{longtblr}[
    caption = {Update User Information Test},
    label = {tblr:update_user_information},
  ]{
    vline{1-3} = {-}{},
    hline{-} = {1-2}{},
    colspec={X[2,l] X[5,l]},
  }
  \textbf{Description} & \textbf{Update user information} \\
  \textbf{Purpose} & {
    Test whether the user can update information successfully.
  } \\
  \textbf{Precondition} & {
    User at the user profile page.
  } \\
  \textbf{Steps} & {
    1. Click the “Cập nhật thông tin” button.
    \\2. System shows the update information form.
    \\3. Fill in all the fields: First Name, Last Name, Email, Phone Number, Address.
    \\4. Click the submit button.
  } \\
  \textbf{Expected result} & {
    Account information is changed and displayed on the user profile page.
  } \\
  \textbf{Execution results} & {
    PASSED
  } \\
\end{longtblr}


\begin{longtblr}[
    caption = {Update User Information with Missing Last Name Test},
    label = {tblr:update_user_information_missing_last_name},
  ]{
    vline{1-3} = {-}{},
    hline{-} = {1-2}{},
    colspec={X[2,l] X[5,l]},
  }
  \textbf{Description} & \textbf{Update user information with missing field Last Name} \\
  \textbf{Purpose} & {
    Test whether the user can update information with a missing field.
  } \\
  \textbf{Precondition} & {
    User at the user profile page.
  } \\
  \textbf{Steps} & {
    1. Click the “Cập nhật thông tin” button.
    \\2. System shows the update information form.
    \\3. Fill in the fields: Last Name, Email, Phone Number, Address. Leave the field First Name empty.
    \\4. Click the submit button.
  } \\
  \textbf{Expected result} & {
    Submit form failed, system alerted user.
  } \\
  \textbf{Execution results} & {
    PASSED
  } \\
\end{longtblr}


The same test was done with other fields: last name, email, phone number, address; which also yielded the same results.

\subsubsection*{Give away pet}

\begin{longtblr}[
    caption = {Give Away Pet Test},
    label = {tblr:give_away_pet},
  ]{
    vline{1-3} = {-}{},
    hline{-} = {1-2}{},
    colspec={X[2,l] X[5,l]},
  }
  \textbf{Description} & \textbf{Give away pet} \\
  \textbf{Purpose} & {
    Test whether the user can give away a pet successfully.
  } \\
  \textbf{Precondition} & {
    User at the give pet page.
    \\ User is authenticated.
  } \\
  \textbf{Steps} & {
    1. System shows the give pet form. The first page is the pet image upload.
    \\2. Click the image upload zone. The system shows the upload window.
    \\3. Choose a .jpg image and click OK. The system shows the uploaded picture in the preview row.
    \\4. Click the “Tiếp tục” button. The system shows the pet information page.
    \\5. Fill in the pet information fields.
    \\6. Click the “Tiếp tục” button. The system shows pet giveaway terms.
    \\7. Tick the accept checkbox.
    \\8. Click the submit form button.
  } \\
  \textbf{Expected result} & {
    Submit pet giveaway form successfully.
  } \\
  \textbf{Execution results} & {
    PASSED
  } \\
\end{longtblr}


\begin{longtblr}[
    caption = {Submit Give Away Pet Form Without Uploading Pet Picture Test},
    label = {tblr:submit_give_away_pet_without_picture},
  ]{
    vline{1-3} = {-}{},
    hline{-} = {1-2}{},
    colspec={X[2,l] X[5,l]},
  }
  \textbf{Description} & \textbf{Submit give away pet form without uploading pet picture} \\
  \textbf{Purpose} & {
    Test whether the user can submit the give away pet form without uploading a picture of the pet.
  } \\
  \textbf{Precondition} & {
    User at the give pet page.
    \\ User is authenticated.
  } \\
  \textbf{Steps} & {
    1. System shows the give pet form. The first page is the pet image upload.
    \\2. Click the “Tiếp tục” button. System shows the pet information page.
    \\3. Fill in the pet information fields.
    \\4. Click the “Tiếp tục” button. System shows pet giveaway terms.
    \\5. Tick the accept checkbox.
    \\6. Click the submit form button.
  } \\
  \textbf{Expected result} & {
    Submit pet giveaway form fails. System alerts the user.
  } \\
  \textbf{Execution results} & {
    PASSED
  } \\
\end{longtblr}


\begin{longtblr}[
    caption = {Submit Give Away Pet Form Without Pet Information Test},
    label = {tblr:submit_give_away_pet_without_info},
  ]{
    vline{1-3} = {-}{},
    hline{-} = {1-2}{},
    colspec={X[2,l] X[5,l]},
  }
  \textbf{Description} & \textbf{Submit give away pet form without pet information} \\
  \textbf{Purpose} & {
    Test whether the user can submit the give away pet form without giving the information of the pet.
  } \\
  \textbf{Precondition} & {
    User at the give pet page.
    \\ User is authenticated.
  } \\
  \textbf{Steps} & {
    1. System shows the give pet form. The first page is the pet image upload.
    \\2. Click the image upload zone. System shows the upload window.
    \\3. Choose a .jpg image and click OK. The system shows the uploaded picture in the preview row.
    \\4. Click the “Tiếp tục” button. System shows the pet information page.
    \\5. Click the “Tiếp tục” button. System shows pet giveaway terms.
    \\6. Tick the accept checkbox.
    \\7. Click the submit form button.
  } \\
  \textbf{Expected result} & {
    Submit pet giveaway form fails. System alerts the user.
  } \\
  \textbf{Execution results} & {
    PASSED
  } \\
\end{longtblr}


\begin{longtblr}[
    caption = {Submit Give Away Pet Form Without Reading Give Away Terms Test},
    label = {tblr:submit_give_away_pet_without_terms},
  ]{
    vline{1-3} = {-}{},
    hline{-} = {1-2}{},
    colspec={X[2,l] X[5,l]},
  }
  \textbf{Description} & \textbf{Submit give away pet form without reading give away terms} \\
  \textbf{Purpose} & {
    Test whether the user can submit the give away pet form without reading the terms.
  } \\
  \textbf{Precondition} & {
    User at the give pet page.
    \\ User is authenticated.
  } \\
  \textbf{Steps} & {
    1. System shows the give pet form. The first page is the pet image upload.
    \\2. Click the “Tiếp tục” button. System shows the pet information page.
    \\3. Click the “Tiếp tục” button. System shows pet giveaway terms.
    \\4. Try to tick the accept checkbox.
  } \\
  \textbf{Expected result} & {
    Cannot tick the accept checkbox. System alerts the user.
  } \\
  \textbf{Execution results} & {
    PASSED
  } \\
\end{longtblr}


\subsubsection*{Adopt Pet}

\begin{longtblr}[
    caption = {Adopt Pet Test},
    label = {tblr:adopt_pet},
  ]{
    vline{1-3} = {-}{},
    hline{-} = {1-2}{},
    colspec={X[2,l] X[5,l]},
  }
  \textbf{Description} & \textbf{Adopt pet} \\
  \textbf{Purpose} & {
    Test whether the user can submit an adopt application successfully.
  } \\
  \textbf{Precondition} & {
    User at the search pet page.
    \\ User is authenticated.
  } \\
  \textbf{Steps} & {
    1. Click the first pet card on the pet list.
    \\2. System directs to the pet detail page.
    \\3. Click the “Nhận nuôi” button.
    \\4. System shows the adopt application form.
    \\5. Fill in all required information: name, email, phone number, address, home type, adopt time, note, pets owned.
    \\6. Click the submit button.
  } \\
  \textbf{Expected result} & {
    User submits adopt application successfully.
  } \\
  \textbf{Execution results} & {
    PASSED
  } \\
\end{longtblr}


\begin{longtblr}[
    caption = {Submit Adopt Pet on Already Submitted Pet Test},
    label = {tblr:submit_adopt_pet_again},
  ]{
    vline{1-3} = {-}{},
    hline{-} = {1-2}{},
    colspec={X[2,l] X[5,l]},
  }
  \textbf{Description} & \textbf{Submit adopt pet on already submitted pet} \\
  \textbf{Purpose} & {
    Test whether the user can submit an adopt application again on the same pet.
  } \\
  \textbf{Precondition} & {
    User at the search pet page.
    \\ User is authenticated.
    \\ Already done the Adopt pet test case.
  } \\
  \textbf{Steps} & {
    1. Click the first pet card on the pet list.
    \\2. System directs to the pet detail page.
    \\3. Click the “Nhận nuôi” button.
  } \\
  \textbf{Expected result} & {
    System alerts the user that they already sent the adopt application for this pet.
  } \\
  \textbf{Execution results} & {
    PASSED
  } \\
\end{longtblr}


\begin{longtblr}[
    caption = {Adopt Owned Pet Test},
    label = {tblr:adopt_owned_pet},
  ]{
    vline{1-3} = {-}{},
    hline{-} = {1-2}{},
    colspec={X[2,l] X[5,l]},
  }
  \textbf{Description} & \textbf{Adopt owned pet} \\
  \textbf{Purpose} & {
    Test whether the user can submit an adopt application for a pet that has already been adopted.
  } \\
  \textbf{Precondition} & {
    User is authenticated.
    \\ System has owned pet.
  } \\
  \textbf{Steps} & {
    1. Use the owned pet ID in the test data to visit the pet detail page by adding “/pet/[pet-id]” to the website URL.
    \\2. System directs to the pet detail page.
  } \\
  \textbf{Expected result} & {
    The “Nhận nuôi” button is not available.
  } \\
  \textbf{Execution results} & {
    PASSED
  } \\
\end{longtblr}


\subsubsection*{Blog}

\begin{longtblr}[
    caption = {Blog Browsing - Blog Card Linking Test},
    label = {tblr:blog_card_linking},
  ]{
    vline{1-3} = {-}{},
    hline{-} = {1-2}{},
    colspec={X[2,l] X[5,l]},
  }
  \textbf{Description} & \textbf{Blog browsing - Blog card linking} \\
  \textbf{Purpose} & {
    Test whether the blog card directs to the correct blog.
  } \\
  \textbf{Precondition} & {
    User at the blog page.
  } \\
  \textbf{Steps} & {
    1. Click the first blog card.
    \\2. System directs to the blog detail page.
  } \\
  \textbf{Expected result} & {
    System displays the blog detail page related to the chosen blog card.
  } \\
  \textbf{Execution results} & {
    PASSED
  } \\
\end{longtblr}


\begin{longtblr}[
    caption = {Blog Browsing - Filter Blog by Health Category Test},
    label = {tblr:filter_blog_by_health_category},
  ]{
    vline{1-3} = {-}{},
    hline{-} = {1-2}{},
    colspec={X[2,l] X[5,l]},
  }
  \textbf{Description} & \textbf{Blog browsing - Filter blog by health category} \\
  \textbf{Purpose} & {
    Test whether the blog cards will be filtered by the category bar.
  } \\
  \textbf{Precondition} & {
    User at the blog page.
  } \\
  \textbf{Steps} & {
    1. Click the “Sức khỏe” option in the category bar.
    \\2. System displays new filtered blog cards.
  } \\
  \textbf{Expected result} & {
    System only displays the blogs about health.
  } \\
  \textbf{Execution results} & {
    PASSED
  } \\
\end{longtblr}


The same test was done with other categories: art, training, and product; which also yielded the same results.

\begin{longtblr}[
    caption = {Blog Create Test},
    label = {tblr:blog_create},
  ]{
    vline{1-3} = {-}{},
    hline{-} = {1-2}{},
    colspec={X[2,l] X[5,l]},
  }
  \textbf{Description} & \textbf{Blog create} \\
  \textbf{Purpose} & {
    Test whether the user can create a blog.
  } \\
  \textbf{Precondition} & {
    User is logged in using an organization account.
    \\ User is at the user profile page.
  } \\
  \textbf{Steps} & {
    1. Click the blog tab. System displays the blog tab content.
    \\2. Click the create blog card.
    \\3. System directs to the blog create page.
    \\4. Fill in the required fields: title, excerpt, blog image, blog content.
    \\5. Click the submit button.
  } \\
  \textbf{Expected result} & {
    User creates the blog successfully.
  } \\
  \textbf{Execution results} & {
    PASSED
  } \\
\end{longtblr}


\begin{longtblr}[
    caption = {Blog Update Test},
    label = {tblr:blog_update},
  ]{
    vline{1-3} = {-}{},
    hline{-} = {1-2}{},
    colspec={X[2,l] X[5,l]},
  }
  \textbf{Description} & \textbf{Blog Update} \\
  \textbf{Purpose} & {
    Test whether the user can update a created blog.
  } \\
  \textbf{Precondition} & {
    User is logged in using an organization account.
    \\ User is at the user profile page.
    \\ Have done the previous Blog Create Test case.
  } \\
  \textbf{Steps} & {
    1. Click the blog tab. System displays the blog tab content.
    \\2. Click the edit button on the created blog.
    \\3. System opens the blog update form.
    \\4. Change the blog title, content, excerpt, image.
    \\5. Click the submit button.
  } \\
  \textbf{Expected result} & {
    User updates the blog successfully.
  } \\
  \textbf{Execution results} & {
    PASSED
  } \\
\end{longtblr}


\section{Responsive testing}
Responsive testing is a crucial aspect of web development, ensuring that
websites are accessible and user-friendly across different devices and
screen sizes. By testing the website's responsiveness, we could verify
that it displayed correctly on various devices, including desktops,
laptops, tablets, and smartphones. This testing process involved
evaluating the website's layout, design, and functionality on different
devices, ensuring a consistent and seamless user experience. By
identifying and addressing any responsiveness issues, we could enhance
the website's usability and accessibility, catering to a broader audience
and improving user satisfaction.

Here is the checklist for responsive testing:
\begin{longtblr}[
    caption = {Responsive Testing Checklist},
    label = {tblr:responsive_testing},
  ]{
    hlines, vlines,
    colspec = {X[2,l] X[4,l] X[4,l]},
    row{1} = {font=\bfseries},
  }
  Category                                & Checklist Item                                       & Notes                                        \\
  \SetCell[r=3]{t}Devices                 & Test on mobile phones                                & Common devices: iPhone, Android, etc.        \\
                                          & Test on tablets                                      & Common devices: iPad, Android tablets, etc.  \\
                                          & Test on laptops and desktops                         & Different screen resolutions                 \\
  \SetCell[r=4]{c}Screen Sizes            & Test on small screens (e.g., 320px width)            & Minimum screen width                         \\
                                          & Test on medium screens (e.g., 768px width)           & Common tablet screen width                   \\
                                          & Test on large screens (e.g., 1024px width and above) & Common laptop/desktop screen widths          \\
                                          & Test on extra-large screens (e.g., 1920px width)     & Large desktop/monitor screen widths          \\
  \SetCell[r=2]{c}Orientation             & Test in portrait mode                                & Vertical screen orientation                  \\
                                          & Test in landscape mode                               & Horizontal screen orientation                \\
  \SetCell[r=5]{c}Browsers                & Test on Chrome                                       & Most widely used browser                     \\
                                          & Test on Firefox                                      & Popular alternative browser                  \\
                                          & Test on Safari                                       & Default browser on iOS devices               \\
                                          & Test on Edge                                         & Default browser on Windows devices           \\
                                          & Test on Opera                                        & Another popular browser option               \\
  \SetCell[r=2]{c}Performance             & Check page load speed                                & Tools: Google PageSpeed Insights, GTmetrix   \\
                                          & Check image optimization                             & Ensure images are responsive and compressed  \\
                                          & Test navigation menus                                & Ensure menus work and are accessible         \\
  \SetCell[r=4]{c}Functionality           & Test forms (input fields, buttons, etc.)             & Ensure all forms are functional              \\
                                          & Test buttons and links                               & Ensure they are easily clickable             \\
                                          & Test media (videos, sliders, etc.)                   & Ensure media is responsive and playable      \\
                                          & Test interactive elements (e.g., accordions, modals) & Ensure they work as expected                 \\
  \SetCell[r=3]{c}Content                 & Check text readability                               & Ensure text is readable at all sizes         \\
                                          & Check image and video scaling                        & Ensure media scales properly                 \\
                                          & Check layout consistency                             & Ensure layout remains consistent             \\
  \SetCell[r=3]{c}Accessibility           & Test keyboard navigation                             & Ensure site can be navigated with a keyboard \\
                                          & Check for screen reader compatibility                & Ensure compatibility with screen readers     \\
                                          & Check color contrast and text size                   & Ensure sufficient contrast and readability   \\
  \SetCell[r=5]{c, valign=t}Miscellaneous & Test touch gestures                                  & Ensure touch gestures work (e.g., swiping)   \\
                                          & Check for any horizontal scrolling                   & Avoid horizontal scrolling                   \\
                                          & Test for viewport meta tag                           & Ensure correct viewport settings             \\
                                          & Check for broken links                               & Ensure no broken links                       \\
                                          & Test social media sharing buttons                    & Ensure they work and are properly positioned \\
\end{longtblr}

\section{Performance testing}

\section{Load testing}